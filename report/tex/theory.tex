%!TEX root = /Users/Daniel/Documents/Imperial/project/tevatron-higgs/report/report.tex
The standard model represents our current understanding of fundamental particles and their interactions. Its development stemmed from the discovery in the 1970s of a symmetry between the weak and electromagnetic forces which allowed them to be unified as a single force. The equations of this unified theory correctly described the electroweak force and its associated bosons, but with a serious issue that all emerged without mass. Whilst true for a photon, we know that the W and Z bosons have significant mass\cite{CERN:aa}.

To solve this problem, a mechanism was proposed by theorists Robert Brout, François Englert and Peter Higgs, whereby particles are given mass through interaction with the Higgs field. Mass of a particle is dependent on the strength of its interaction with the field – for example photons do not interact with it at all. The particle associated with the fundamental Higgs field is the Higgs boson\cite{CERN:aa}.

Observation of a new particle with a mass of 126 Gev was announced on 4 July 2012 by the ATLAS and CMS experiments at CERN's Large Hadron Collider. This particle is thought to be consistent with the Higgs Boson, and lead to the award of the Nobel prize in 2013 to François Englert and Peter Higgs.


Quantum field theories such as the standard model comprise two basic types of field, fermions that lead to matter particles and bosons that lead to force-carrier. The spin-statistics theorem asserts that these two types are differentiated in that bosons have integer spins and fermions half-odd-integer spins. The creation of a symmetry that relates fermions and bosons forms the basis supersymmetry, whose major proponent is the existence of a new set of partner particles to those in the standard model. The superpartners differ by a spin of a half, so that bosons will be partnered with fermions and vice-versa\cite{Neil-Lambert:aa}. 

The existence of these new particles could solve a major problem of fixing the mass of the Higgs Boson. The observed mass of the Higgs boson is puzzlingly light when interactions with the particles of the standard model should make it very heavy. The addition of the extra particles in supersymmetry could cancel out contributions to the Higgs mass from their Standard-Model partners. 

The minimal supersymmetric standard model (MSSM) investigated here is the supersymmetry model with the minimum number of new particle states and interactions that are consistent with phenomenology\cite{baer2006weak}. Whereas in the Standard Model the Higgs field is introduced as a weak isospin doublet, in the
MSSM the presence of two of these doublets leads to five physical Higgs bosons, three neutral (collectively denoted as $\phi$): h, H, and A ; and two charged: $H^-$ and $H^+$ \cite{Abazov201197}.

To describe the MSSM Higgs model, two free parameters are conventionally chosen - $\tan\beta$, the ratio of the vacuum expectation values of the two Higgs doublets, and the mass of A, one of the MSSM Higgs. Although $\tan\beta$ is a free parameter in MSSM, a large value of $\tan\beta \approx 35$ would explain the ratio of the top to bottom quark mass, and also provide good reason for the observed density of dark matter.

The value of $\tan\beta$ provides an enhancement factor for the couplings of the Higgs to fermions in the SM compared to the MSSM. The couplings of the Higgs in the MSSM are proportional to those in the SM and depend also on the type of quark and Higgs. For large $\tan\beta$, two of the Higgs Bosons A and either h or H have approximately the same mass. Whilst their coupling to down-type quarks is enhanced by $\tan\beta$ compared to the SM, their coupling to up-type quarks is reduced.  Due to this enhancement, the most probable mode of decay for the three neutral Higgs bosons is $\phi  \rightarrow b\bar{b} $  with a branching ratio near 90\% (the next most probable being 
$\phi  \rightarrow \tau\bar{\tau} $ ). Due to large multijet backgrounds a direct search for $\phi \rightarrow b\bar{b} $ is difficult. Instead, the case $\phi b \rightarrow b\bar{b}b $ where $\phi$ is produced in association with one b quark is used which.
